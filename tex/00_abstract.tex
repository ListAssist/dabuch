
\selectlanguage{english}
The diploma project „ListAssist -- Die intelligente Einkaufliste“ is an app, 
which functions like a more advanced shopping list.
The idea is that our project should replace the conventional shopping list. 
With the help of artificial intelligence, our 
product is able to analyze the consumer behavior of the users 
and automatically generate new shopping lists. 
The product also has a group function, where you can share your 
shopping lists with your family or friends. 
The statistics show which products are bought most frequently. 

The user can change the shopping lists by removing and deleting 
products as usual. New lists can also be 
added or removed easily. The shopping receipts can get scanned in via 
smartphone and based on the image the products are filtered and 
saved in the app.

To understand how the app works and its features, an animated video
is available to explain the product.
\selectlanguage{ngerman}

% Im englischen Abstract sollte inhaltlich das Gleiche stehen wie in
% der deutschen Kurzfassung. Versuchen Sie daher, die Kurzfassung präzise
% umzusetzen, ohne aber dabei Wort für Wort zu übersetzen. Beachten
% Sie bei der Übersetzung, dass gewisse Redewendungen aus dem Deutschen
% im Englischen kein Pendant haben oder völlig anders formuliert werden
% müssen und dass die Satzstellung im Englischen sich (bekanntlich)
% vom Deutschen stark unterscheidet. Es empfiehlt sich übrigens – auch
% bei höchstem Vertrauen in die persönlichen Englischkenntnisse – eine
% kundige Person für das „proof reading“ zu engagieren. Die richtige
% Übersetzung für „Diplomarbeit“ ist übrigens schlicht thesis, allenfalls
% „diploma thesis“ oder „Master’s thesis“, auf keinen Fall aber „diploma
% work“ oder gar „dissertation“\citep{hagenberg}.

% Wichtig ist wegen des Abteilens ein \code{\textbackslash{}begin\{english\}}
% bzw. \code{\textbackslash{}selectlanguage\{ngerman\}}.
