\hypertarget{hauptziele}{%
\section{Hauptziele}\label{hauptziele}}

RE-M 1 Dokumentation

Es wird eine Dokumentation in Form eines Diplomarbeitsbuchs geführt, in
welcher der Fortschritt, der angefallene Aufwand und die Testergebnisse
dokumentiert werden.

RE-M 2 Corporate Identity

Das Design der Applikation ist mit der Projektwebsite und anderen
Grafiken einheitlich.

RE-M 3 Design

Die Applikation wird nach den Material-Design Richtlinien designed und
ist dann damit auch eine Material-App.

RE-M 4 Einkaufsliste

Eine Einkaufsliste kann erstellt, bearbeitet und abgearbeitet werden.

RE-M 5 Automatisches einlesen der Rechnung per Foto

Die Rechnung kann abfotografiert werden, wodurch Produkte automatisch
mit künstlicher Intelligenz erkannt werden und von der ausgewählten
Einkaufsliste gestrichen werden oder eine ganz neue erstellt wird.

RE-M 6 Verlauf der Einkäufe

Der Verlauf der Einkäufe und Rechnungen kann im Nachhinein angesehen
werden.

RE-M 7 Sicherung der Daten

Die Daten (Einkaufslisten, Gruppen, Benutzer, Statistiken) werden auf
Firebase bzw. in Firestore gespeichert.

RE-M 8 Automatische Einkaufsliste

Sobald genug Daten zur Verfügung stehen, werden automatisch, an den
Benutzer angepasste, Einkaufslisten erstellt.

RE-M 9 Authentifizierung

Der Benutzer kann sich in der Applikation mittels externer Plattformen
wie Google, Facebook oder Twitter anmelden bzw. registrieren. Weiters
wird die Möglichkeit geboten sich mit einer eigenen E-Mail und Passwort
zu registrieren und einzuloggen.

RE-M 10 Gruppen

Benutzer können Gruppen erstellen, in denen sie gemeinsame
Einkaufslisten erstellen, bearbeiten und abarbeiten können.

RE-M 11 Einladungen

Benutzer können andere Nutzer der App über ihre E-Mail Adresse zu einer
Gruppe einladen.

RE-M 12 Projektwebsite

Es gibt eine Projektwebsite, auf der das Projekt vorgestellt wird. Auf
der Seite befinden sich die Features unseres Produktes, der
Diplomarbeitsantrag und eine Beschreibung des Teams.

RE-M 13 Einstellungen

Benutzer können in den Einstellungen ihren Nutzernamen, Passwort und
E-Mail ändern.

RE-M 14 Animationsvideo

Es wird ein Animationsvideo, welches das Produkt beschreibt, produziert.

RE-M 15 Produktdatensammlung

Um den Usern ein einfaches Bedienen der App zu ermöglichen, werden Daten
über Produkte bereits vorgelegt. Diese entstehen aus selbst erstellten
Daten oder einer vorgegeben Datenbank eines größeren Geschäftes.

RE-M 16 Veröffentlichung

Die App wird im Google PlayStore veröffentlicht.

\hypertarget{optionale-ziele}{%
\section{Optionale Ziele}\label{optionale-ziele}}

RE-O 1 Darstellung von gesammelten Daten

Die Daten, welche vom Benutzer zur Verfügung gestellt werden, sind in
Form von Statistiken angezeigt. Ein Kreisdiagramm sowie ein
Balkendiagramm stellen die Verteilung der Ausgaben auf den einzelnen
Kategorien dar.

RE-O 2 Benachrichtigung bei Angeboten

Die App zeigt dem Benutzer aktuelle Angebote von Geschäften, passend zu
seinen Einkaufslisten, an.

RE-O 3 Rechnungen und Einkauflisten als PDF exportieren

Die App zeigt dem Benutzer aktuelle Angebote von Geschäften, passend zu
seinen Einkaufslisten, an.

\hypertarget{nicht-ziele}{%
\section{NICHT Ziele}\label{nicht-ziele}}

RE-N 1 Veröffentlichung

Die App wird im AppStore zur Verfügung stehen.

\hypertarget{individuelle-aufgabenstellungen-der-teammitglieder}{%
\section{Individuelle Aufgabenstellungen der
Teammitglieder}\label{individuelle-aufgabenstellungen-der-teammitglieder}}

\hypertarget{rafael-doja}{%
\subsection{Rafael Doja}\label{rafael-doja}}

\begin{longtable}[]{@{}ll@{}}
\toprule
Themenschwerpunkt & Scrum Master, Design \& Gestaltung des
Produktes\tabularnewline
\midrule
\endhead
Ziele und Anforderungen & RE-M 1 Dokumentation\tabularnewline
& RE-M 2 Corporate Identity\tabularnewline
& RE-M 3 Design\tabularnewline
& RE-M 14 Animationsvideo\tabularnewline
& RE-M 15 Produktdatensammlung\tabularnewline
\bottomrule
\end{longtable}

\hypertarget{filip-coja}{%
\subsection{Filip Coja}\label{filip-coja}}

\begin{longtable}[]{@{}ll@{}}
\toprule
\begin{minipage}[b]{0.31\columnwidth}\raggedright
Themenschwerpunkt\strut
\end{minipage} & \begin{minipage}[b]{0.63\columnwidth}\raggedright
Programmierung der App\strut
\end{minipage}\tabularnewline
\midrule
\endhead
\begin{minipage}[t]{0.31\columnwidth}\raggedright
Ziele und Anforderungen\strut
\end{minipage} & \begin{minipage}[t]{0.63\columnwidth}\raggedright
RE-M 5 Automatisches einlesen der Rechnung per Foto\strut
\end{minipage}\tabularnewline
\begin{minipage}[t]{0.31\columnwidth}\raggedright
\strut
\end{minipage} & \begin{minipage}[t]{0.63\columnwidth}\raggedright
RE-M 7 Sicherung der Daten\strut
\end{minipage}\tabularnewline
\begin{minipage}[t]{0.31\columnwidth}\raggedright
\strut
\end{minipage} & \begin{minipage}[t]{0.63\columnwidth}\raggedright
RE-M 8 Automatische Einkaufsliste\strut
\end{minipage}\tabularnewline
\begin{minipage}[t]{0.31\columnwidth}\raggedright
\strut
\end{minipage} & \begin{minipage}[t]{0.63\columnwidth}\raggedright
RE-M 9 Authentifizierung\strut
\end{minipage}\tabularnewline
\begin{minipage}[t]{0.31\columnwidth}\raggedright
\strut
\end{minipage} & \begin{minipage}[t]{0.63\columnwidth}\raggedright
RE-M 12 Projektwebsite\strut
\end{minipage}\tabularnewline
\begin{minipage}[t]{0.31\columnwidth}\raggedright
\strut
\end{minipage} & \begin{minipage}[t]{0.63\columnwidth}\raggedright
RE-M 16 Veröffentlichung\strut
\end{minipage}\tabularnewline
\begin{minipage}[t]{0.31\columnwidth}\raggedright
\strut
\end{minipage} & \begin{minipage}[t]{0.63\columnwidth}\raggedright
RE-O 3 Rechnungen und Einkauflisten als PDF exportieren\strut
\end{minipage}\tabularnewline
\bottomrule
\end{longtable}

\hypertarget{tobias-seczer}{%
\subsection{Tobias Seczer}\label{tobias-seczer}}

\begin{longtable}[]{@{}ll@{}}
\toprule
Themenschwerpunkt & Programmierung der App\tabularnewline
\midrule
\endhead
Ziele und Anforderungen & RE-M 4 Einkaufsliste\tabularnewline
& RE-M 7 Sicherung der Daten\tabularnewline
& RE-M 8 Automatische Einkaufsliste\tabularnewline
& RE-M 10 Gruppen\tabularnewline
& RE-M 11 Einladungen\tabularnewline
\bottomrule
\end{longtable}

\hypertarget{daniel-kisling}{%
\subsection{Daniel Kisling}\label{daniel-kisling}}

\begin{longtable}[]{@{}ll@{}}
\toprule
Themenschwerpunkt & Programmierung der App\tabularnewline
\midrule
\endhead
Ziele und Anforderungen & RE-M 4 Einkaufsliste\tabularnewline
& RE-M 6 Verlauf der Einkäufe\tabularnewline
& RE-M 13 Einstellungen\tabularnewline
& RE-M 15 Produktdatensammlung\tabularnewline
& RE-O 1 Darstellung von gesammelten Daten\tabularnewline
& RE-O 2 Benachrichtingungen bei Angeboten\tabularnewline
\bottomrule
\end{longtable}
