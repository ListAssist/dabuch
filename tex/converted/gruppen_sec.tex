\hypertarget{allgemein}{%
\section{Allgemein}\label{allgemein}}

\hypertarget{einkaufslisten-in-gruppen}{%
\section{Einkaufslisten in Gruppen}\label{einkaufslisten-in-gruppen}}

Da bereits für die Einkaufslisten Widgets erstellt wurden, wurden diese
einfach wiederverwendet. Das erspart sowohl die erneute Programmierung
des Widgets als auch Probleme mit der Einheitlichkeit, da man bei einer
Änderung beide Widgets wieder anpassen müsste.

\hypertarget{uxfcbersicht-der-benutzer-in-einer-gruppe}{%
\section{Übersicht der Benutzer in einer
Gruppe}\label{uxfcbersicht-der-benutzer-in-einer-gruppe}}

Die Liste der Benutzer einer Gruppe ist ein Array von Maps, die den
Namen, eine UID und eine URL für das Profilbild enthalten. Mit der
``map'' methode werden aus den einzelnen Maps Widgets erstellt. Die
Darstellung ist ein einfaches ``Row'' Widget mit einem Profilbild vorne,
und dem Namen nach dem Profilbild. Falls ein Benutzer keine URL für ein
Profilbild besitzt oder während das Bild lädt, wird ein selbst
erstelltes Standard Profilbild angezeigt. Da der Besitzer der Gruppe
ebenfalls in der Liste der Mitglieder ist, wird der Besitzer vor dem
``mappen'' entfernt und er bekommt eine etwas andere Darstellung,
nämlich mit einem grünen ``Gruppenersteller'' nach seinem Namen. Sollte
der Name zu lang sein um ihn richtig darzustellen, wird mittels der
Einstellung \passthrough{\lstinline!overflow: TextOverflow.ellipsis!}
der Text gekürzt und mit \ldots{} am Ende dargestellt.

\hypertarget{probleme}{%
\section{Probleme}\label{probleme}}
